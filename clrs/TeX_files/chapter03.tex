\chapter{Growth of Functions}
\section{Asymptotic notation}

\qn{3.1-1} Let $f(n)$ and $g(n)$ be asymptotically nonnegative functions. Using the basic definition of $\Theta$-notation, prove that $\max(f(n), g(n)) = \Theta(f(n) + g(n))$.

\sol \qed

\qn{3.1-2}

Show that for any real constants $a$ and $b$, where $b > 0$, $$(n+a)^b = \Theta(n^b).$$

\sol \qed

\qn{3.1-3} Explain why the statement, ``The running time of the algorithm $A$ is at least $O(n^2)$'' is meaningless.

\sol \qed

\qn{3.1-4} Is $2^{n+1} = O(2^n)$? Is $2^{2n} = O(2^n)$?

\sol \qed

\qn{3.1-5} Prove Theorem 3.1.

\pf \qed

\qn{3.1-6} Prove that the running time of an algorithm is $\Theta(g(n))$ if and only if its worst-case running time is $O(g(n))$ and its best-case running time is $\Omega(g(n))$.

\pf \qed

\qn{3.1-7} Prove that $o(g(n)) \cap \omega(g(n))$ is the empty set.

\pf \qed

\qn{3.1-8} We can extend our notation to the case of two parameters $n$ and $m$ that can go to infinity independently at different rates. For a given function $g(n,m)$, we denote by $O(g(n,m))$ the set of functions
\begin{align*}
    O(g(n,m)) = \{f(n,m) \,  : \, & \text{there exist positive constants } c, n_0, \text{ and } m_0 \\
    & \text{such that } 0 \leq f(n,m) \leq cg(n,m) \\
    & \text{for all }n \geq n_0 \text{ or } m \geq m_0\}.
\end{align*}
Give corresponding definitions for $\Omega(g(n,m))$ and $\Theta(g(n,m))$.

\sol \qed

\section{Standard notations and common functions}

\qn{3.2-1} Show that if $f(n)$ and $g(n)$ are monotonically increasing functions, then so are the functions $f(n) + g(n)$ and $f(g(n))$, and if $f(n)$ and $g(n)$ are in addition nonnegative, then $f(n) \cdot g(n)$ is monotonically increasing.

\sol \qed

\qn{3.2-2} Prove equation (3.16)

\pf \qed

\qn{3.2-3} Prove equation (3.19). Also prove that $n! = \omega(2^n)$ and $n! = o(n^n)$.

\pf \qed

\qn{3.2-4 *} Is the function $\lceil \lg n \rceil !$ polynomially bounded? Is the function $\lceil \lg \lg n \rceil !$ polynomially bounded?

\sol \qed

\qn{3.2-5 *} Which is asymptotically larger: $\lg(\lg * n)$ or $\lg * (\lg n)$

\sol \qed

\qn{3.2-6} Show that the golden ratio $\phi$ and its conjugate $\hat{\phi}$ both satisfy the equation $x^2 = x + 1$.

\sol \qed

\qn{3.2-7} Prove by induction that the $i$th Fibonacci number satisfies the equality $$F_i = \frac{\phi^i - \hat{\phi}^i}{\sqrt{5}},$$ where $\phi$ is the golden ratio and $\hat{\phi}$ is its conjugate.

\pf \qed

\qn{3.2-8} Show that $k \ln k = \Theta(n)$ implies $k = \Theta(n / \ln n)$.

\sol \qed

\section{Problems}

\qn{3-1} \textbf{Asymptotic behavior of polynomials}

Let $$p(n) = \sum_{i=0}^d a_i n^i,$$ where $a_d > 0$, be a degree-$d$ polynomial in $n$, and let $k$ be a constant. Use the definitions of the asymptotic notations to prove the following properties

\begin{enumerate}[(a)]
    \item If $k \geq d$, then $p(n) = O(n^k)$.
    \item If $k \leq d$, then $p(n) = \Omega(n^k)$.
    \item If $k = d$, then $p(n) = \Theta(n^k)$.
    \item If $k > d$, then $p(n) = o(n^k)$.
    \item If $k < d$, then $p(n) = \omega(n^k)$.
\end{enumerate}

\sol \qed

\qn{3-2} \textbf{Relative asymptotic growths}

Indicate, for each pair of expressions $(A,B)$ in the table below, whether $A$ is $O, o, \Omega, \omega$ or $\Theta$ of $B$. Assume that $k \geq 1, \, \varepsilon > 0$, and $c > 1$ are constants. Your answer should be in the form of the table with ``yes'' or ``no'' written in each box.

\,

\begin{center}
\begin{tabular}{|l|l|l|l|l|l|l|}
    \hline
    A           & B                 & $O$ & $o$ & $\Omega$ & $\omega$ & $\Theta$ \\\hline
    $\lg^k n$   & $n^{\varepsilon}$ &     &     &          &          &                         \\\hline
    $n^k$       & $c^n$             &     &     &          &          &                         \\\hline
    $\sqrt{n}$  & $n^{\sin n}$      &     &     &          &          &                         \\\hline
    $2^n$       & $2^{n/2}$         &     &     &          &          &                         \\\hline
    $n^{\lg c}$ & $c^{\lg n}$       &     &     &          &          &                         \\\hline
    $\lg(n!)$   & $\lg(n^n)$        &     &     &          &          &                        \\\hline
\end{tabular}
\end{center}
\qed

\qn{3-3} \textbf{Ordering by asymptotic growth rates}
\begin{enumerate}[(a)]
    \item Rank the following functions by order of growth; that is, find an arrangement $g_1, g_2, \dots, g_{30}$ of the functions satisfying $g_1 = \Omega(g_2)$, $g_2 = \Omega(g_3), \dots, g_{29} = \Theta(g_{30})$. Partition your list into equivalence classes such that functions $f(n)$ and $g(n)$ are in the same class iff $f(n) = \Theta(g(n))$.
    
\begin{center}
    \begin{tabular}{llllll}
        $\lg(\lg^* n)$     & $2^{\lg^* n}$         & $(\sqrt{2})^{\lg n}$ & $n^2$           & $n!$      & $(\lg n)!$     \\
        $(\frac{3}{2})^n$ & $n^3$                & $\lg^2 n$            & $\lg(n!)$       & $2^{2^n}$ & $n^{1/\lg n}$  \\
        $\ln \ln n$       & $\lg^* n$             & $n\cdot 2^n$         & $n^{\lg \lg n}$ & $\ln n$   & 1              \\
        $2^{\lg n}$       & $(\lg n)^{\lg n}$    & $e^n$                & $4^{\lg n}$     & $(n+1)!$  & $\sqrt{\lg n}$ \\
        $\lg^* (\lg n)$    & $2^{\sqrt{2 \lg n}}$ & $n$                  & $2^n$           & $n \lg n$ & $2^{2^{n+1}}$ 
        \end{tabular}
\end{center}

    \item Give an example of a single nonnegative function $f(n)$ such that for all functions $g_i(n)$ in part (a), $f(n)$ is neither $O(g_i(n))$ nor $\Omega(g_i(n))$.
\end{enumerate}

\sol \qed

\qn{3-4} \textbf{Asymptotic notation properties}

Let $f(n)$ and $g(n)$ be asymptotically positive functions. Prove or disprove each of the following conjectures.

\begin{enumerate}
    \item $f(n) = O(g(n))$ implies $g(n) = O(f(n))$.
    \item $f(n) + g(n) = \Theta(\min(f(n), g(n)))$.
    \item $f(n) = O(g(n))$ implies $\lg(f(n)) = O(\lg(g(n)))$, where $\lg(g(n)) \geq 1$ and $f(n) \geq 1$ for all sufficiently large $n$.
    \item $f(n) = O(g(n))$ implies $2^{f(n)} = O(2^{g(n)})$.
    \item $f(n) = O((f(n))^2)$.
    \item $f(n) = O(g(n))$ implies $g(n) = \Omega(f(n))$.
    \item $f(n) = \Theta(f(n/2))$
    \item $f(n) + o(f(n)) = \Theta(f(n))$.
\end{enumerate}

\sol \qed

\qn{3-5} \textbf{Variations on $O$ and $\Omega$}

Some authors define $\Omega$ in a slightly different way than we do; let's use $\Omega_\infty$ (read ``omega infinity'') for this alternative definition. We say that $f(n) = \Omega_\infty(g(n))$ if there exists a positive constant $c$ such that $f(n) \geq cg(n) \geq 0$ for infinitely many positive integers $n$.

\begin{enumerate}[(a)]
    \item Show that for any two functions $f(n)$ and $g(n)$ that are asymptotically nonnegative, either $f(n) = O(g(n))$ or $f(n) = \Omega_\infty (g(n))$ or both, whereas this is not true if we use $\Omega$ in place of $\Omega_\infty$.
    \item Describe the potential advantages and disadvantages of using $\Omega_\infty$ instead of $\Omega$ to characterize the running times of programs.
\end{enumerate}

Some authors also definte $O$ in a slightly different manner; let's use $O'$ for the alternative definition. We say that $f(n) = O'(g(n))$ iff $|f(n)| = O(g(n))$.

\begin{enumerate}[(a)]
    \setcounter{enumi}{2}
    \item What happens to each direction of the ``if and only if'' in Theorem 3.1 if we substitute $O'$ for $O$ but still use $\Omega$?
\end{enumerate}

Some authors define $\tilde{O}$ (read ``soft-oh'') to mean $O$ with logarithmic factors ignored:

\begin{align*}
\tilde{O}(g(n)) = \{f(n) : \, & \text{there exist positive constants } c, \, k, \text{ and } n_0 \text{ such that} \\
& 0 \leq f(n) \leq cg(n) \lg^k (n) \text{ for all } n \geq n_0\}.
\end{align*}

\begin{enumerate}[(a)]
    \setcounter{enumi}{3}
    \item Definte $\tilde{\Omega}$ and $\tilde{\Theta}$ in a similar manner. Prove the corresponding analog to Theorem 3.1.
\end{enumerate}

\sol \qed

\qn{3-6} \textbf{Iterated functions}

We can apply the iteration operator $^*$ used in the $\lg^*$ function to any monotonically increasing function $f(n)$ over the reals. For a given constant $c \in \R$, we define the iterated function $f_c^*$ by $$f_c^*(n) = \min\{i \geq 0 \, : \, f^{(i)}(n) \leq c\},$$ which need not be well defined in all cases. In other words, the quantity $f_c^*(n)$ is the number of iterated applications of the function $f$ required to reduce its argument down to $c$ or less.

For each of the following functions $f(n)$ and constants $c$, give as tight of a bound as possible on $f_c^*(n).$

\begin{enumerate}[(a)]
    \item $f(n) = n-1, \, c=0$.
    \item $f(n) = \lg n, \, c=1$.
    \item $f(n) = n/2, \, c=1$.
    \item $f(n) = n/2, \, c=2$.
    \item $f(n) = \sqrt{n}, \, c=2$.
    \item $f(n) = \sqrt{n}, \, c=1$.
    \item $f(n) = n^{1/3}, \, c=2$.
    \item $f(n) = n/\lg n, \, n=2$.
\end{enumerate}

\sol \qed