\chapter{Getting Started}
\section{Insertion sort}

\qn{1.} Using Figure 2.2 as a model, illustrate the operation of \verb|INSERTION-SORT| on the array $A = \langle 31, 41, 59, 26, 41, 58\rangle$.

\sol First pass: key set to 41. \verb|A[1] = 31 < 41|, so the while loop breaks, and the second element stays put.

Second pass: key set to 59. Same steps as above.

Third pass: Key set to 26. \verb|A[3] = 59 > 26|, so set \verb|A[4] = 59|. \verb|A[2] =| \verb|41 > 26|, so set \verb|A[3] = 41|. \verb|A[1] = 31 > 26|, so set \verb|A[2] = 31|. Loop breaks, set \verb|A[1] = 26|.

Fourth pass: Key set to 41. \verb|A[4] = 59 > 41|, so set \verb|A[5] = 59|. \verb|A[3] =| \verb|41|, so the while loop breaks. Set \verb|[4] = 41|.

Final pass: Key set to 58. \verb|A[5] = 59 > 58|, so set \verb|A[6] = 59|. \verb|A[4] =| \verb|41 < 58|, so the while loop breaks. Set \verb|[5] = 58|. The final sorted array is $\langle 26, 31, 41, 41, 58, 59\rangle$.\qed

\qn{2.} Rewrite the \verb|INSERTION-SORT| procedure to sort into nonincreasing instead of nondecreasing order.

\sol Rewrite the algorithm as follows.
\begin{Verbatim}[frame=single,numbers=left,samepage=true,label=REVERSE-INSERTION-SORT(A)]
for j = 2 to A.length
    key = A[j]
    // Insert A[j] into the sorted sequence A[1..j-1].
    i = j - 1
    while i > 0 and A[i] < key // The only change
        A[i + 1] = A[i]
        i = i - 1
    A[i + 1] = key
\end{Verbatim}

The implementation of this can be found in \newline\verb|code/c/chapter02/reverse_insertion_sort/main.c|.\qed

\qn{3.} Consider the \textbf{searching problem}:

\textbf{Input:} A sequence of $n$ numbers $A = \langle a_1, a_2, \dots, a_n \rangle$ and a value $v$.

\textbf{Output:} An index $i$ such that $v = A[i]$ or the special value \verb|NIL| if $v$ does not appear in $A$.

Write pseudocode for \textbf{linear search}, which scans through the sequence, looking for $v$. Using a loop invariant, prove that your algorithm is correct. Make sure that your loop invariant fulfills the three necessary properties.

\sol \qed

\qn{4.} Consider the problem of adding two $n$-bit binary integers, stored in two $n$-element arrays $A$ and $B$. The sum of the two integers should be stored in binary form in an $(n+1)$-element array $C$. State the problem formally and write pseudocode for adding the two integers.

\sol \qed