\chapter{Getting Started}
\section{Insertion sort}

\qn{1.} Using Figure 2.2 as a model, illustrate the operation of \verb|INSERTION-SORT| on the array $A = \langle 31, 41, 59, 26, 41, 58\rangle$.

\sol \qed

\qn{2.} Rewrite the \verb|INSERTION-SORT| procedure to sort into nonincreasing instead of nondecreasing order.

\sol \qed

\qn{3.} Consider the \textbf{searching problem}:

\textbf{Input:} A sequence of $n$ numbers $A = \langle a_1, a_2, \dots, a_n \rangle$ and a value $v$.

\textbf{Output:} An index $i$ such that $v = A[i]$ or the special value \verb|NIL| if $v$ does not appear in $A$.

Write pseudocode for \textbf{linear search}, which scans through the sequence, looking for $v$. Using a loop invariant, prove that your algorithm is correct. Make sure that your loop invariant fulfills the three necessary properties.

\sol \qed

\qn{4.} Consider the problem of adding two $n$-bit binary integers, stored in two $n$-element arrays $A$ and $B$. The sum of the two integers should be stored in binary form in an $(n+1)$-element array $C$. State the problem formally and write pseudocode for adding the two integers.

\sol \qed