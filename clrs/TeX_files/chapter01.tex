\chapter{The Role of Algorithms in Computing}
\section{Algorithms}

\qn{1.} Give a real-world example that requires sorting or a real-world example that requires computing a convex hull.

\sol Any sort of problem involving organizing probably has sorting. For example, ordering a set of people by any quantitative characteristic (height, weight, test scores, age, their one mile run time, etc.).

Convex hulls are used widely in graphical simulations, for example to detect collisions. \qed

\qn{2.} Other than speed, what other measures of efficiency might one use in a real-world setting?

\sol Computer memory. Programmer time (time to implement algorithm). Energy efficiency. \qed

\qn{3.} Select a data structure that you have seen previously, and discuss its strengths and limitations.

\sol Linked lists have the advantage of being extendible up to the memory limit of your computer. Disadvantage of needing to traverse entire list to find an element (or determine if one isn't there) \qed

\qn{4.} How are the shortest-path and traveling-salesman problems given above similar? How are they different?

\sol Both are problems on weighted graphs involving attempting to minimize some subset of the graph (a path or a cycle). However, the shortest-path problem makes no requirement to end at the starting point, whereas the traveling-salesman problem does. This makes the shortest-path problem in P (solved by several algorithms, such as Dijkstra's), and the traveling-salesman in NP-complete. \qed

\qn{5.} Come up with a real-world problem in which only the best solution will do. Then come up with one in which a solution that is ``approximately'' the best is good enough.

\sol A problem where only the best solution will do is one that requires a fully correct solution, like needing a sorted list for a directory. One where an approximate solution, as discussed earlier in the chapter, would be solving traveling-salesman. \qed

\section{Algorithms as a technology}
\qn{1.} Give an example of an application that requires algorithmic content at the application level, and discuss the function of the algorithms involved.

\sol Google Maps when giving directions to users uses algorithms to determine the quickest route, factoring in distance, tolls, predicted traffic at the time the user would be at a certain road, any reported accidents, etc. \qed

\qn{2.} Suppose we are comparing implementations of insertion sort and merge sort on the same machine. For inputs of size $n$, insertion sort runes in $8n^2$ steps, while merge sort runs in $64n \lg n$ steps. For which values of $n$ does insertion sort beat merge sort?

\sol This is asking for what values of $n$ we have that $8n^2 \leq 64 n \lg n$. We can simplify this to $2^{n/8} \leq n$. Plugging in values yields the largest valid integer value of $n$ being 43. \qed

\qn{3.} What is the smallest value of $n$ such that an algorithm whose running time is $100n^2$ runs faster than an algorithm whose running time is $2^n$ on the same machine?

\sol This is asking what the smallest value of $n$ such that $100n^2 \leq 2^n$ is, which is equivalent to when $n \leq \frac{2^{n/2}}{10}$. The value of $n$ is 15. \qed

\section{Problems}
\qn{1.} For each function $f(n)$ and time $t$ in the following table, determine the largest size $n$ of a problem that can be solved in time $t$, assuming that the algorithm to solve the problem takes $f(n)$ microseconds.

\sol \,

\,

\begingroup
\scriptsize
\begin{tabular}{|l|l|l|l|l|l|l|l|}
    \hline
               & 1 second  & 1 minute    & 1 hour       & 1 day         & 1 month         & 1 year                                                      & 1 century          \\ \hline
    $\lg n$    & $2^{1e6}$ & $2^{6e7}$  & $2^{7.2e9}$ & $2^{8.64e10}$ & $2^{2.59e12}$ & $2^{3.15e13}$ & $2^{3.15e15}$ \\ \hline
    $\sqrt{n}$ & $1e12$    & $3.6e15$    & $1.29e19$   & $7.46e21$   & $6.71e24$   & $9.67e26$                                               & $9.67e30$      \\ \hline
    $n$        & $1e6$     & $6e7$       & $3.6e9$      & $8.64e10$     & $2.59e12$      & $3.1104e13$                                                 & $3.11e15$        \\ \hline
    $n \lg n$  & 62746     & $2.80e6$ & $1.33e8$  & $2.75e9$   & $7.18e10$    & $7.8709e11$                                                 & $6.76e13$       \\ \hline
    $n^2$      & 1000      & 7745        & 60000        & 293938        & 1609968         & 5577096                                                     & 55770960           \\ \hline
    $n^3$      & 100       & 391         & 1532         & 4420          & 13736           & 98169                                                       & 455661             \\ \hline
    $2^n$      & 19        & 25          & 31           & 36            & 41              & 44                                                          & 51                 \\ \hline
    $n!$       & 9         & 11          & 12           & 13            & 15              & 17                                                          & 18                 \\ \hline
\end{tabular}
\endgroup

\qed